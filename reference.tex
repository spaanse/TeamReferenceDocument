\documentclass[8pt,a4paper,landscape,oneside]{amsart}
\usepackage[T1]{fontenc}
\usepackage[utf8]{inputenc}
\usepackage[scaled]{beramono}
\usepackage{fullpage}
\usepackage[top=3pt, bottom=1cm, left=0.3cm, right=0.3cm]{geometry}
\usepackage{fancyhdr}
\usepackage{titling}
\usepackage{multicol}
\usepackage{minted}
\usepackage{lipsum}
\usepackage{enumitem}

\title{Team Reference Document}
\author{Jelmer Firet}

\setminted{autogobble,tabsize=2,fontsize=\normalsize,escapeinside=@@}

\fancyhf{}
\pagestyle{fancy}
\lhead{Team Reference Document}
\rhead{\thepage}
\cfoot{\;}
\setlength{\headheight}{15.2pt}
\setlength{\droptitle}{-30pt}
% \setlength{\columnseprule}{0.4pt}
\setlength{\columnsep}{3pt}
\posttitle{\par\end{center}}
\renewcommand{\headrulewidth}{0.4pt}
\renewcommand{\footrulewidth}{0.4pt}

\newenvironment{itemize_compact}
{ \begin{itemize}[leftmargin=.5cm]
    \setlength{\itemsep}{0pt}
    \setlength{\parskip}{0pt}
    \setlength{\parsep}{0pt}     }
{ \end{itemize}                  }

\begin{document}
	\begin{multicols*}{2}
		\maketitle
		\vspace*{-45pt}
		\thispagestyle{fancy}
		\begin{multicols}{2}
			\tableofcontents
			\columnbreak
			\vfill\null
		\end{multicols}
		\vfill\null
		\columnbreak
		\section{Setup} % (fold)
		\begin{multicols*}{2}
			\subsection{\texttt{\textasciitilde/check.sh}}
			\inputminted{sh}{_code/setup/check.sh}
			\subsection{\texttt{\textasciitilde/.vimrc}}
			\inputminted{sh}{_code/setup/vimrc.sh}
			\columnbreak
			\subsection{\texttt{C++} header}
			\inputminted{cpp}{_code/setup/header.cpp}
			\vfill\null
		\end{multicols*}
		\subsection{Checking copied TRD code} Most lines of code have hash codes. Use \texttt{\textasciitilde/check.sh} to calculate these.\\
		The hash of a line is $h(\text{line})=h(\text{prev line})\oplus\text{md5}(\text{line without whitespace})$.
		For the first line of TRD code $h(\text{prev line})$ is $00$, so you need to guarantee that in the copied code. Suppose the current hash of the previous line is \texttt{ac}. Then look in row \texttt{a} and column \texttt{c} to find \texttt{69H}. A new line with \texttt{//69H} will then have hash \texttt{00}. To skip over sections of TRD code: compute the xor of the actual and expected hash of the previous line and insert the corresponding comment. The first char in each cell is the xor of the row and column, so $a\oplus c=6$\\
		\begin{center}		
			\texttt{
			\begin{tabular}{|p{0.1cm}|*{4}{*{4}{c}|}}
			\hline
			&0&1&2&3&4&5&6&7&8&9&a&b&c&d&e&f\\
			\hline
			0&008&11p&218&3E5&48Y&52N&62G&76r&84M&91G&a1m&b5A&c15&d2h&e2F&f17\\
			1&12w&00Y&3A5&284&52J&426&79L&64x&93f&80e&b5Y&a4i&d18&c3e&fc5&e8s\\
			2&28F&38I&05h&1ak&64d&74r&47r&52i&a26&b3p&81u&9ee&eAy&f2E&caH&d7L\\
			3&31f&29A&11J&02k&72R&65w&5ih&413&bAH&a7L&9i3&85H&f8Y&e2M&dci&c3D\\
			\hline
			4&43c&546&62L&77d&03e&10N&287&32c&c0L&dej&e4c&fdp&82J&927&a1e&bd5\\
			5&53R&401&72h&6E3&12r&0bb&36A&209&d2T&c8z&f52&e6c&95k&88G&bAv&a1E\\
			6&63m&757&40c&50b&2aM&360&01v&1GF&ecx&f3q&cAL&d36&a09&b1d&824&91D\\
			7&70b&6aa&537&402&32J&22c&146&0bL&f14&e3G&d0k&cbJ&b5a&ab9&94c&83H\\
			\hline
			8&80F&9ae&a06&b8G&c9n&d1i&e2a&faL&075&17z&20G&3D0&4aN&5At&62d&715\\
			9&92b&8di&bb2&a2H&d4i&c32&fcL&eJa&13d&01D&34q&23a&54d&48p&7ce&60y\\
			a&a7f&b7y&81R&93t&e8b&f0k&c5t&d6f&25J&355&047&14i&69H&7fL&486&57p\\
			b&b2w&a0F&94b&810&f5e&e14&d75&c2j&33b&22D&11h&05d&72n&61d&58A&40z\\
			\hline
			c&cii&dMn&e1z&f2u&83f&94E&aFr&b1a&4a2&5s0&642&75v&08y&12E&2ai&3E8\\
			d&d2N&c08&fE6&e1N&91L&885&b1i&a2A&538&434&79c&643&16a&02E&3Ae&25T\\
			e&e0v&f0J&cct&d6z&a0N&b7r&817&953&6dD&727&43v&525&22e&31E&05c&12p\\
			f&f5r&e4A&d6x&c9x&b1y&aek&90z&82s&70m&625&54t&444&34h&201&13T&031\\
			\hline
			\end{tabular}
			}
		\end{center}
		\vfill\null
	\end{multicols*}
	\newpage
	\begin{multicols*}{4}		
		\section{Datastructures}
		\subsection{Union-Find}
		\inputminted[firstline=9,lastline=19]{c++}{_code/datastructures/union_find.cpp}
		\subsection{Segment Tree}
		\inputminted[firstline=14,lastline=52]{c++}{_code/datastructures/segment_tree.cpp}
		\vfill\null
		\columnbreak
		\subsection{Treap}
		\inputminted[firstline=16,lastline=67]{c++}{_code/datastructures/treap.cpp}
		\vfill\null
		\columnbreak
		\inputminted[firstline=68,lastline=115]{c++}{_code/datastructures/treap.cpp}
		\noindent
		Variations of the treap code can result in
		\begin{itemize_compact}
			\item max heap: \mintinline{c++}{h(n)= n->v}, remove h from node.
			\item min heap: \mintinline{c++}{h(n)=-n->v}, remove h from node.
			\item cartesian tree: let \mintinline{c++}{n->v=idx} and \mintinline{c++}{n->h=val}.
		\end{itemize_compact}
		\vfill\null
		\columnbreak
		\subsection{Implicit Treap}
		\inputminted[firstline=16,lastline=69]{c++}{_code/datastructures/implicit_treap.cpp}
		\vfill\null
	\end{multicols*}
	\newpage
	\begin{multicols*}{4}
		\subsection{Fenwick Tree}
		\inputminted[firstline=14,lastline=32]{c++}{_code/datastructures/fenwick_tree.cpp}
		\subsection{Sparse Table}
		\inputminted[firstline=15,lastline=23]{c++}{_code/datastructures/sparse_table.cpp}
		\vfill\null
	\end{multicols*}
\end{document}
